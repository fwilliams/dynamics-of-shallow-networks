\documentclass{article}
\usepackage{graphicx}

%% Useful packages
\usepackage{amsmath, amsthm, amssymb}
\usepackage{graphicx}
\usepackage[colorinlistoftodos]{todonotes}
\usepackage[colorlinks=true, allcolors=blue]{hyperref}
\usepackage{tikz}
\usepackage{dsfont}


%% Language and font encodings
\usepackage[english]{babel}
\usepackage[utf8x]{inputenc}
\usepackage[T1]{fontenc}

%% Sets page size and margins
\usepackage[top=1.25in, bottom=1.5in, left=1.5in, right=1.5in]{geometry}
\setlength\parindent{0pt}

%% Math environments
\newtheorem{theorem}{Theorem}
%\numberwithin{theorem}{section}
\newtheorem{proposition}[theorem]{Proposition}
\newtheorem{lemma}[theorem]{Lemma}
\newtheorem{corollary}[theorem]{Corollary}
\newtheorem{definition}[theorem]{Definition}
\newtheorem{problem}[theorem]{Problem}
\newtheorem{remark}[theorem]{Remark}
\newtheorem{example}[theorem]{Example}
\newtheorem{conjecture}[theorem]{Conjecture}
\newtheorem{algorithm}[theorem]{Algorithm}

\newcommand{\RP}{\mathbb{RP}}
\newcommand{\RR}{\mathbb{R}}
\newcommand{\QQ}{\mathbb{Q}}
\newcommand{\PP}{\mathbb{P}}
\newcommand{\CC}{\mathbb{C}}
\newcommand{\ZZ}{\mathbb{Z}}
\newcommand{\NN}{\mathbb{N}}

\newcommand{\eg}{{\em e.g.}}
\newcommand{\ie}{{\em i.e.}}

\def\note#1{\textcolor{blue}{#1}}

\title{{\bf Spline kernels}}
\author{}
\date{}

\begin{document}
\maketitle


We consider a function of the form

\begin{equation}\label{eq:4-1-1}
f(x) = \int_D \varphi(x,\theta) c(\theta) d\theta.
\end{equation}

We are interested in least-squares interpolation, \ie, in the problem

\begin{equation}\label{eq:4-1-min}
\min \int_D \|c(\theta)\|^2 d\theta \quad \mbox{ s.t. } f(x_i) = y_i, \,\, i=0,\ldots,n.
\end{equation}

There is a linear map
\begin{equation}
M: L^2(D) \rightarrow \RR^{n+1}, \qquad c(\theta) \mapsto \left(\int_D c(\theta) \varphi(x_i,\theta) d\theta \right)_{i=0,\ldots,n}.
\end{equation}

Assuming that $y = (y_i)_{i=0,\ldots,n}$ lies in the image of this map, we have that 

\begin{equation}
    \hat c(\theta) = M^* (MM^*)^{-1} y,
\end{equation}
where $M^*$ denotes the adjoint operator. 

\begin{lemma}
The adjoint operator $M^*$ is represented by the map
\begin{equation}
M^*: \RR^{n+1} \rightarrow L^2(D), \qquad y \mapsto \sum_{i=0}^n y_i \varphi(x_i,\cdot)
\end{equation}
In particular, $MM^*$ is represented by the Gram matrix
\begin{equation}
K_{ij} = \int_D \varphi(x_i,\theta) \varphi(x_j,\theta) d\theta \in \RR^{(n+1)\times(n+1)}.
\end{equation}
\end{lemma}
\begin{proof} It is enough to note that
\begin{equation}
\langle M(c), y \rangle_{\RR^{n+1}} = \sum_{i=0}^n \int_D y_i \varphi(x_i, \theta) c(\theta) d\theta = \left \langle \sum_{i=0}^n y_i \varphi(x_i,\cdot), c(\cdot)  \right \rangle_{L^2(D)}
\end{equation}
\end{proof}

We deduce that~\eqref{eq:4-1-min} can be written as
\begin{equation}
    \hat{c}(\theta) = M^* (K)^{-1} y = \sum_{i=0}^n v_i \varphi(x_i,\cdot), \mbox{ where } v = (v_i) = K^{-1} y.
\end{equation}

The final function is

\begin{equation}
    \hat f(x) = \int_D \hat c(\theta) \varphi(x,\theta) d\theta = \sum_{i=0}^n v_i \int_D \varphi(x,\theta)\varphi(x_i,\theta) d\theta = \sum_{i=0}^n v_i K(x,x_i).
\end{equation}

Assuming $x\le y$, we have that

\begin{equation}
\begin{aligned}
\int_a^b [x-s]_+[y-s]_+ ds &= \int_a^{\min\{x,y\}} (x-s)(y-s) ds =
\left[\frac{1}{3} \, s^{3} - \frac{1}{2} \, s^{2} {\left(x + y\right)} + s x y\right]_a^{\min\{x,y\}}\\
&=-\frac{1}{6} \, {\left(2 \, a + x - 3 \, y\right)} {\left(a - x\right)}^{2}
\end{aligned}
\end{equation}


\begin{equation}
\begin{aligned}
\int_a^b [x-s]_+[y-s]_+ ds &+ \int_a^b [-x-s]_+[-y-s]_+ ds = \int_a^{x} (x-s)(y-s) ds 
+ \int_y^{b} (s-x)(s-y) ds \\
&=-\frac{1}{6} \, {\left(2 \, a + x - 3 \, y\right)} {\left(a - x\right)}^{2} + \frac{1}{6} \, {\left(2 \, b - 3 \, x + y\right)} {\left(b - y\right)}^{2}
\end{aligned}
\end{equation}

The two kernels are:

\[
K_1(x_i, x_j) = \left[\frac{1}{6} \, {\left(3 \, x_j -2 \, a - x_i\right)} {\left(x_i-a\right)}^{2}\right], \qquad x_i \le x_j.
\]

\[
K_2(x_i, x_j) = \left[\frac{1}{6} \, {\left(3 \, x_j - 2 \, a - x_i \right)} {\left(x_i - a\right)}^{2} + \frac{1}{6} \, {\left(2 \, b - 3 \, x_i + x_j\right)} {\left(b - x_j\right)}^{2} \right], \qquad x_i \le x_j.
\]


\end{document}